% IEEE Access-style paper for "Machine Learning-Driven Research Paper Retrieval System"
\documentclass{ieeecolor}
\usepackage{generic}
\usepackage{amsmath,amssymb}
\usepackage{graphicx}
\usepackage{booktabs}
\usepackage{listings}
\usepackage{xcolor}
\usepackage{hyperref}

\def\header{Machine Learning-Driven Research Paper Retrieval System}

\begin{document}

\title{Machine Learning-Driven Research Paper Retrieval System}
\author{Author Name1, Author Name2% <-keep this line for spacing
\thanks{Corresponding author: Author Name1 (email: author1@example.com).}}

\markboth{IEEE Access, ~Vol.~X, No.~Y, Month~2025}{\header}
\maketitle

\begin{abstract}
This paper presents a research paper retrieval system that integrates machine learning (ML) and natural language processing (NLP) to overcome the limitations of traditional keyword-based search. We propose an end-to-end pipeline that ingests scholarly metadata and abstracts, performs text preprocessing, and applies both lexical and semantic models to rank results. The system leverages TF-IDF for lexical relevance and Sentence Transformer embeddings for semantic similarity, combined via a hybrid ranking strategy. A modern frontend enables natural language queries with advanced filters, while a lightweight backend orchestrates model inference and database retrieval. Experimental results on a curated dataset demonstrate improved precision, recall, and MAP over keyword-only baselines.
\end{abstract}

\begin{keywords}
Machine Learning, Natural Language Processing, Information Retrieval, Semantic Search, Recommendation System
\end{keywords}

\section{Introduction}
We combine TF-IDF and Sentence Transformer embeddings to improve literature retrieval relevance and robustness to synonymy.

\section{Related Work}
Brief review of digital libraries (Scholar, Semantic Scholar, IEEE Xplore), TF-IDF/BM25, BERT-based semantic search, and hybrid methods.

\section{Methodology}
Dataset construction from titles/abstracts with licensing considerations; preprocessing (tokenization, stopwords, optional stemming); models (TF-IDF + all-MiniLM-L6-v2); hybrid fusion with tuned $\alpha$.

\section{Implementation}
\subsection{API and Storage}
`/search` request: `{"query": str, "filters": {..}, "topk": int, "search_type": "tfidf|semantic|hybrid"}`; response: list of `{id, score}`. Storage: SQLite/MongoDB; embeddings cached.

\subsection{SQL DDL (Excerpt)}
\begin{lstlisting}[basicstyle=\ttfamily\footnotesize]
CREATE TABLE papers (
 id TEXT PRIMARY KEY, title TEXT, abstract TEXT,
 authors TEXT, year INTEGER, venue TEXT, vector BLOB);
\end{lstlisting}

\subsection{Evaluation Script (Pseudo-code)}
\begin{lstlisting}[language=Python,basicstyle=\ttfamily\footnotesize]
for q in queries:
  ranked = search(q.text)
  p.append(precision_at_k(ranked, gold[q.id], 10))
  r.append(recall_at_k(ranked, gold[q.id], 10))
  m.append(average_precision(ranked, gold[q.id]))
  n.append(ndcg_at_k(ranked, gold[q.id], 10))
report(mean(p), mean(r), mean(m), mean(n))
\end{lstlisting}

\section{Results and Discussion}
Placeholders for P@10, R@10, MAP, nDCG@10, ablation across $\alpha$.

\section{Threats to Validity and Reproducibility}
Bias in labels and domain coverage; mitigations and fixed seeds/frozen deps.

\section{Conclusion and Future Work}
Personalization, multilingual retrieval, summarization, and scalable vector DBs.

\bibliographystyle{IEEEtran}
\bibliography{references}

\end{document}
